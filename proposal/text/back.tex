\section{Non Volatile Memory Express}

1 . Peformance Analysis of NVMe SSDs and their Implication on Real World Databases

Xu et al~\cite{XXX} showed that NVMe-backed
database applications deliver up to 8X
superior client-side performance over enterprise-class,
SATA-based SSDs and presented a detailed, quantitative
analysis of all the factors contributing to the low-latency,
high-throughput characteristic of NVMe drivers, including
system software stack.

Some Usefule Facts:
1) In recent years, a large amount of the world's compute and 
storage has been pushed onto back-end datacenters.
There are multitude of applications, millions of active users
and billions of daily transactions. The load on the I/O subsystem
has been increasing at an alarming rate.

2) To meet ever-increasing performance demands, storage subsystem
and data-storage devices had to evolve. While every new generation
of drives provided a modest performance improvement over its predecesor,
several innovations in drive technology introduced major leafs over
the previous state-of-the-art. Solid State Drivers (SSDs) were
the first technology leap -- owing to the lack of moving parts,
they delivered significant better random-access read performance
as compared to hard disk drivers (HDDs). NVMe represents the
second major technology leap. NVMe is software based standard
that was specifically optimized for SSDs connected through the PCIe
interface. The effort has been focused towards building a standard
that will not only provide better performance and scalability,
but will also be flexible enough to accommodate storage solutions
based on memory technology of the future. Designed with
those goals in mind, NVMe-based driver are able to deliver
vastly superior performance in terms of both bandwidth
and latency. 

3) Where the performance benefits of NVMe-based
drives emanate from? 






2. Implementing a High-Performance Key-Value Store Using a Trie of B+-Trees with Cursors



