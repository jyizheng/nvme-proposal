%\documentclass{sig-alternate}
\documentclass[10pt,twocolumn,conference]{IEEEtran}

%\usepackage[paper=letterpaper, top=1.0in, bottom=1.0in, left=1.0in, right=1.0in]{geometry}

\usepackage{caption}
\usepackage{times}   % Enable for producing camera ready's
\usepackage{subfigure}
\usepackage{color}
\usepackage{comment}
\usepackage{pslatex}
\usepackage{caption}
\usepackage{amssymb, amsmath}
\usepackage{epsfig}
\usepackage{algorithm}
\usepackage{algorithmic}
\usepackage{multirow}
\usepackage{balance, listings}
\usepackage{url} 
\usepackage{tensor} 


\usepackage{fixltx2e}
\usepackage{xcolor}
\def\SPSB#1#2{\rlap{\textsuperscript{\textcolor{black}{#1}}}\SB{#2}}
\def\SP#1{\textsuperscript{\textcolor{black}{#1}}}
\def\SB#1{\textsubscript{\textcolor{black}{#1}}}


\newcommand{\code}[1]{\texttt{#1}}
\newcommand{\defn}[1]	{{\textit{\textbf{\boldmath #1}}}}


%\usepackage[hyperref,colorlinks]{hyperref}
%\usepackage{hyperref}
%\usepackage{breakurl}

\usepackage{textcomp}
\usepackage{balance, listings}
 
\definecolor{dkgreen}{rgb}{0,0.6,0}
\definecolor{gray}{rgb}{0.5,0.5,0.5}
\definecolor{mauve}{rgb}{0.58,0,0.82}
\lstset{ %
  language=C,                     % the language of the code
  float=[tb],
%  basicstyle=\footnotesize,       % the size of the fonts that are used for the code
%  numbers=left,                  % where to put the line-numbers
%  numberstyle=\tiny\color{gray}, % the style that is used for the line-numbers
  stepnumber=2,                   % the step between two line-numbers. If it's 1, each line
                                  % will be numbered
  numbersep=5pt,                  % how far the line-numbers are from the code
  backgroundcolor=\color{white},  % choose the background color. You must add \usepackage{color}
  showspaces=false,               % show spaces adding particular underscores
  showstringspaces=false,         % underline spaces within strings
  showtabs=false,                 % show tabs within strings adding particular underscores
  frame=tb,                       % adds a frame around the code
%frame=lrb,xleftmargin=\fboxsep,xrightmargin=-\fboxsep
  rulecolor=\color{black},        % if not set, the frame-color may be changed on line-breaks within not-black text (e.g. commens (green here))
  tabsize=2,                      % sets default tabsize to 2 spaces
  captionpos=t,                   % sets the caption-position to bottom
  breaklines=true,                % sets automatic line breaking
  breakatwhitespace=false,        % sets if automatic breaks should only happen at whitespace
  title=\lstname,                 % show the filename of files included with \lstinputlisting;
                                  % also try caption instead of title
  keywordstyle=\color{blue},      % keyword style
  commentstyle=\color{dkgreen},   % comment style
  stringstyle=\color{mauve},      % string literal style
  escapeinside={\%*}{*)},         % if you want to add LaTeX within your code
  morekeywords={*,...},           % if you want to add more keywords to the set
  emph={cudaMalloc,cudaMallocNVRAM},
  emphstyle=\underbar
}
 


\newenvironment{itemizer}{\begin{itemize}
                \setlength{\parsep}{0cm}
                \setlength{\itemsep}{-.4em}}{\end{itemize}}


\makeatletter
\let \@copyrightspace\relax
\newcommand*{\rom}[1]{\expandafter\@slowromancap\romannumeral #1@}
\makeatother

\pagestyle{empty}
\graphicspath{{graphics/}}
\hyphenation{op-tical net-works semi-conduc-tor}

\begin{document}

%%
%% This is file `abstract.sty',
%% generated with the docstrip utility.
%%
%% The original source files were:
%%
%% abstract.dtx  (with options: `usc')
%% 
%%  Copyright 2000 Peter R. Wilson
%% 
%%  This program is provided under the terms of the
%%  LaTeX Project Public License distributed from CTAN
%%  archives in directory macros/latex/base/lppl.txt.
%% 
%% Author: Peter Wilson (CUA)
%%         now at: peter.r.wilson@boeing.com
%% 
\NeedsTeXFormat{LaTeX2e}
\ProvidesPackage{abstract}[2001/02/11 v1.1 configurable abstracts]

\newif\if@bsonecol
  \@bsonecoltrue
\newif\ifadd@bstotoc
  \add@bstotocfalse
\newif\ifnumber@bs
  \number@bsfalse
\newif\if@bsrunin
  \@bsruninfalse

\DeclareOption{original}{\@bsonecolfalse}
\DeclareOption{addtotoc}{\add@bstotoctrue}
\DeclareOption{number}{\number@bstrue}
\DeclareOption{runin}{\@bsrunintrue}
\ProcessOptions\relax
\if@bsrunin\number@bsfalse\fi

\newcommand{\abstractnamefont}{\normalfont\small\bfseries}
\newcommand{\abstracttextfont}{\normalfont\small}

\newcommand{\absnamepos}{center}
\newlength{\abstitleskip} \setlength{\abstitleskip}{-0.5em}
\newlength{\absleftindent}
\newlength{\absrightindent}
\newlength{\absparindent}
\newlength{\absparsep}

\newcommand{\abslabeldelim}[1]{\def\@bslabeldelim{#1}}
\abslabeldelim{}
\newcommand{\@bsrunintitle}{%
  \hspace*{\abstitleskip}{\abstractnamefont\abstractname\@bslabeldelim}}

\if@titlepage
  \setlength{\absleftindent}{\z@}
  \renewcommand{\abstractnamefont}{\normalfont\bfseries}
  \renewcommand{\abstracttextfont}{\normalfont}
  \setlength{\abstitleskip}{0em}
\else
  \if@twocolumn
    \if@bsonecol
      \setlength{\absleftindent}{\leftmargin}
    \else
      \setlength{\absleftindent}{\z@}
      \renewcommand{\abstractnamefont}{\normalfont\Large\bfseries}
      \renewcommand{\abstracttextfont}{\normalfont}
      \renewcommand{\absnamepos}{flushleft}
      \setlength{\abstitleskip}{0em}
    \fi
  \else
    \setlength{\absleftindent}{\leftmargin}
  \fi
\fi
\setlength{\absrightindent}{\absleftindent}
\AtBeginDocument{\setlength{\absparindent}{\parindent}
                 \setlength{\absparsep}{\parskip}}

\newenvironment{@bstr@ctlist}{%
  \list{}{%
          %%\topsep        \z@
          \partopsep     \z@
          \listparindent \absparindent
          \itemindent    \listparindent
          \leftmargin    \absleftindent
          \rightmargin   \absrightindent
          \parsep        \absparsep}%
  \item\relax}
  {\endlist}

\newcommand{\put@bsintoc}{%
  \ifadd@bstotoc
    \ifnumber@bs\else
      \@ifundefined{chapter}{\addcontentsline{toc}{section}{\abstractname}}{%
                             \addcontentsline{toc}{chapter}{\abstractname}}
    \fi
  \fi}

\newcommand{\num@bs}{%
  \@ifundefined{chapter}{\section{\abstractname}}{%
                         \chapter{\abstractname}}
}

\if@titlepage
  \renewenvironment{abstract}{%
    \titlepage
    \null\vfil
    \@beginparpenalty\@lowpenalty
    \if@bsrunin\else
      \ifnumber@bs \num@bs \else
        \begin{\absnamepos}%
          \abstractnamefont \abstractname
          \@endparpenalty\@M
        \end\absnamepos%
        \vspace{\abstitleskip}%
      \fi
    \fi
    \put@bsintoc%
    \begin{@bstr@ctlist}\if@bsrunin\@bsrunintitle\fi\abstracttextfont}%
    {\par\end{@bstr@ctlist}\vfil\null\endtitlepage}
\else
  \renewenvironment{abstract}{%
    \if@bsrunin\else
      \ifnumber@bs \num@bs \else
        \begin{\absnamepos}\abstractnamefont\abstractname\end\absnamepos%
        \vspace{\abstitleskip}%
      \fi
    \fi
    \put@bsintoc%
    \begin{@bstr@ctlist}\if@bsrunin\@bsrunintitle\fi\abstracttextfont}%
    {\par\end{@bstr@ctlist}}
\fi

\newenvironment{onecolabstract}{%
  \begin{@twocolumnfalse}\begin{abstract}}{%
  \end{abstract}\end{@twocolumnfalse}}

\providecommand{\appendiargdef}[2]{\begingroup
  \toks@\expandafter{#1{##1}#2}%
  \edef\@bsx{\endgroup \def\noexpand#1####1{\the\toks@}}%
  \@bsx}

\appendiargdef{\thanks}{%
  \protected@xdef\@bs@thanks{\@bs@thanks
    \protect\footnotetext[\the\c@footnote]{#1}}%
}
\let\@bs@thanks\@empty

\newcommand{\saythanks}{\begingroup
  \renewcommand{\thefootnote}{\fnsymbol{footnote}}\@bs@thanks
  \endgroup\global\let\@bs@thanks\@empty}

\endinput
%%
%% End of file `abstract.sty'.

\input{text/intro}
\section{Write-Optimized Data Structures}

This is theory part.

\input{text/conclusion}

%\bibliographystyle{latex8}
%\bibliographystyle{plain}
\bibliographystyle{abbrv}
\bibliography{mem.bib}
\end{document}
